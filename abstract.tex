This research will investigate the concern of software execution reproducibility.
Reproducibility in that sense denotes three aspects that can be assessed. Firstly dicussing the very
reproducibile aspect of a software is certainly one of them. Secondly laying out a standard structure and
 terminology that will be the base to represent and reproduce an execution context. And finally
 macthing between an original context and a reproduced one to diccuss certain similarity metrics regarding
the ouputs, meaning also how far away is the reproduced ouput from the original one and associate that to
the context environmental change issues.

In this research we tend more to lean toward the second and then the third aspect of this area.
Currently there are some interests on that field because it becomes obvious for one doing more or 
less any research (simulation, experimentations, etc...) computer related that having a way to preserve
the state of a trial for various reasons (sharing, publication, knowledge by difference or similarity, ...)
will be more conveniant that using current old fashion solutions (Lab note books, Text editors, etc...).

Existing tools like sumatra aret trying to provide a solution to that requirement.
One may ask what is the conciliation part between those citied aspects. Well, one thing is to be
able to have some records of software execution contexts and another thing is to be able to feature
a way to compute that execution and correlate the results with the original context.

Our final goal is to formalise the way of automating these two main aspects we are interested in and 
have this model implemented along with what already exists in Sumatra as a main case study and also
some possible secondary case studies.
