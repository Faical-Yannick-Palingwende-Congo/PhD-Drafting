
+++ Software execution context reproducibility +++
The MGI main goals are to reduce the time and the ressources needed to bring new materials to the market
by innovating in four differents areas: Computational tools, Experimental tools, Collaborative networks
and Digital Data. Software execution context reproducibility expand to all those areas and more. It is certainly data type agnostic
which include digital data and put as simply as one can, it is a computational tool to store execution contexts for
 reproducibility, sharing (which is certainly a purpose in a collaborative way) and the biggest area where domain like reproducibility
has a key value is experiments because whever you fail or succeed an experiement you want to be able reproduce
it as a knowledge in any case or at least go away from the old fashion solutions (lab note books, text editors, etc..)
to keep track of your trials for more adapted solutions (context references for data provenance, context queries, etc..). 
Software execution context reproducibility allow one to continuously have access to the history
of his experiments, simulations, etc.. which save time and resources a lot specially in cases like sharing knowledge,
recovering a research work after some time or lost note books, etc...)
We certainly can add that projects like the NIST data curator align with having these meta data and data files from the execution contexts pushed to the curator also to be accessible and can also be placed in the initiative of data provenance requirement.